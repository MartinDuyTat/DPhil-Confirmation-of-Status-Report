%%%%%%%%%%%%%%%%%%%%%%%%%%%%%%%%%%%%%%%%%%%%%%%%%%%%%%%%%%%%%%%%%
\documentclass[12pt, a4paper, notitlepage, onecolumn]{article}
\usepackage[UKenglish]{babel}                   % UK style
\usepackage[utf8]{inputenc}
\usepackage[margin=1in]{geometry}               % Margin size
\usepackage{hyperref}                           % Coloured hyperlinks
  \hypersetup{colorlinks = true}
\usepackage{lmodern}                            % Modern fonts
\usepackage{graphicx}                           % For figures
\usepackage[percent]{overpic}                   % For figures with text overlay
\usepackage{amsmath,amssymb}                    % Mathematical symbols
\usepackage{mathtools}
\usepackage{siunitx}                            % SI-units
%\sisetup{exponent-product = \cdot}             % Dot product instead of cross product
\sisetup{separate-uncertainty = true}           % Plus-minus uncertainty
\usepackage{physics}                            % Elegant equations in physics
\usepackage{booktabs}                           % Nice lines, for instance in tables
\usepackage[font=small,labelfont=bf]{caption}% Caption
\usepackage{float}                              % Table do not move with [H].
\usepackage{subcaption}                         % For subfigures
\usepackage[en-GB]{datetime2}                   % UK date format
\usepackage{listings}                           %Source code
\usepackage{feynmp}                             % Feynman diagrams
\DeclareGraphicsRule{*}{mps}{*}{}               % Include Feynman diagrams
\usepackage{scalerel}
\newcommand{\mylbrace}[2]{\vspace{#2pt}\hspace{6pt}\scaleleftright[\dimexpr5pt+#1\dimexpr0.06pt]{\lbrace}{\rule[\dimexpr2pt-#1\dimexpr0.5pt]{-4pt}{#1pt}}{.}}
\newcommand{\myrbrace}[2]{\vspace{#2pt}\scaleleftright[\dimexpr5pt+#1\dimexpr0.06pt]{.}{\rule[\dimexpr2pt-#1\dimexpr0.5pt]{-4pt}{#1pt}}{\rbrace}\hspace{6pt}}
\usepackage{xspace}                             % Fancy LHCb symbols
\usepackage{upgreek}
\def\pythia{\mbox{\textsc{Pythia}}\xspace}
\def\evtgen{\mbox{\textsc{EvtGen}}\xspace}
\def\photos{\mbox{\textsc{Photos}}\xspace}
\usepackage{natbib}                      % Set line spacing in references
\setlength{\bibsep}{1.0pt}

\usepackage{titlesec}                    % Set spacing between sections
\titlespacing{\section}{0pt}{\baselineskip}{0.1\baselineskip}
\titlespacing{\subsection}{0pt}{0.1\baselineskip}{0.1\baselineskip}
\titlespacing{\subsubsection}{0pt}{0.1\baselineskip}{0.1\baselineskip}

\usepackage{enumitem}                    % itemize separation

%%%%%%%%%%%%%%%%%%%%%%%%%%%%%%%%%%%%%%%%%%%%%%%%%%%%%%%%%%%%%%%
\title{DPhil thesis outline}
\author{}
\date{}
%\numberwithin{equation}{section}
%%%%%%%%%%%%%%%%%%%%%%%%%%%%%%%%%%%%%%%%%%%%%%%%%%%%%%%%%%%%%%%
\begin{document}
\maketitle
\vspace{-1.5cm}
%%%%%%%%%%%%%%%%%%%%%%%%%%%%%%%%%%%%%%%%%%%%%%%%%%%%%%%%%%%%%%%
\section{Introduction}
\begin{itemize}[nosep]
  \setlength{\itemindent}{0em}
  \item{Introduction to $C\!P$-violation and measurement of $\gamma$ using $b\to c$ transitions}
  \item{Introduction to strong-phase measurements at charm factories}
\end{itemize}

\section{Theoretical background}
  \begin{itemize}[nosep]
    \setlength{\itemindent}{0em}
    \item{Introduction to $C\!P$-violation}
  \end{itemize}
\subsection{The Standard Model and the CKM matrix}
  \begin{itemize}[nosep]
    \setlength{\itemindent}{2em}
    \item[\textendash]{Basic description of the Standard Model}
    \item[\textendash]{Explanation of how $C\!P$-violation arises from the CKM matrix}
  \end{itemize}
\subsection{Measuring \texorpdfstring{$\gamma$}{gamma} using \texorpdfstring{$B^\pm\to Dh^\pm$}{B2Dh} decays}
  \begin{itemize}[nosep]
    \setlength{\itemindent}{2em}
    \item[\textendash]{Theory of $\gamma$ measurements using $B^\pm\to Dh^\pm$ decays}
  \end{itemize}
\subsubsection{Measuring \texorpdfstring{$\gamma$}{gamma} using self-conjugate multibody \texorpdfstring{$D$}{D} final states}
  \begin{itemize}[nosep]
    \setlength{\itemindent}{2.5em}
    \item[\textasteriskcentered]{Description of BPGGSZ and GLW methods}
    \item[\textasteriskcentered]{Motivation for model-independent measurement}
    \item[\textasteriskcentered]{Present yield equations and fit strategy}
  \end{itemize}
\subsubsection{Binning scheme of \texorpdfstring{$D^0\to K^+K^-\pi^+\pi^-$}{D02KKpipi}}
  \begin{itemize}[nosep]
    \setlength{\itemindent}{2.5em}
    \item[\textasteriskcentered]{Strategy for binning of phase space}
  \end{itemize}
\subsection{Strong phases from quantum-correlated \texorpdfstring{$D^0\bar{D^0}$}{D0D0bar} decays}
  \begin{itemize}[nosep]
    \setlength{\itemindent}{2em}
    \item[\textendash]{Theory of quantum-correlated $D^0\bar{D^0}$ processes}
    \item[\textendash]{Present the double tag method and tag types}
    \item[\textendash]{Derive yield equations for the strong-phase fits}
  \end{itemize}

\section{The LHCb experiment}
  \begin{itemize}[nosep]
    \setlength{\itemindent}{0em}
    \item{Description of the LHCb experiment during Run\,$1$ and $2$}
    \item{Focus on VELO, tracking system and RICH}
  \end{itemize}

\section{Analysis of \texorpdfstring{$B^\pm\to[K^+K^-\pi^+\pi^-]_Dh^\pm$}{B2DhD2KKpipi} candidates}
  \begin{itemize}[nosep]
    \setlength{\itemindent}{0em}
    \item{This section contains the whole LHCb analysis}
    \item{Status: Paper submitted to EPJC}
  \end{itemize}
\subsection{Data selection}
  \begin{itemize}[nosep]
    \setlength{\itemindent}{2em}
    \item[\textendash]{Overview of all the selection requirements}
  \end{itemize}
\subsubsection{Initial selection requirements}
  \begin{itemize}[nosep]
    \setlength{\itemindent}{2.5em}
    \item[\textasteriskcentered]{Simple requirements on triggers, mass windows, momentum and RICH}
  \end{itemize}
\subsubsection{Boosted Decision Tree}
  \begin{itemize}[nosep]
    \setlength{\itemindent}{2.5em}
    \item[\textasteriskcentered]{Training and test samples}
    \item[\textasteriskcentered]{Training variables}
    \item[\textasteriskcentered]{Optimisation of working point}
  \end{itemize}
\subsubsection{Final selection requirements}
  \begin{itemize}[nosep]
    \setlength{\itemindent}{2.5em}
    \item[\textasteriskcentered]{Flight significance requirement}
    \item[\textasteriskcentered]{PID requirements}
    \item[\textasteriskcentered]{$K^0_S$ mass veto}
    \item[\textasteriskcentered]{Semileptonic $B^\pm$ decay veto}
    \item[\textasteriskcentered]{Ghost track rejection}
  \end{itemize}
\subsection{Background studies}
  \begin{itemize}[nosep]
    \setlength{\itemindent}{2em}
    \item[\textendash]{For each background, describe the origin, mass shape and yield}
  \end{itemize}
\subsubsection{Charmless background}
  \begin{itemize}[nosep]
    \setlength{\itemindent}{2.5em}
    \item[\textasteriskcentered]{Rejection with flight significance requirements}
    \item[\textasteriskcentered]{Residual background estimate using sidebands}
  \end{itemize}
\subsubsection{Semileptonic \texorpdfstring{$D$}{D} background}
  \begin{itemize}[nosep]
    \setlength{\itemindent}{2.5em}
    \item[\textasteriskcentered]{Overview of all possible decay modes}
    \item[\textasteriskcentered]{RapidSim studies of shape and yield}
  \end{itemize}
\subsubsection{Background from \texorpdfstring{$D^0\to K^-\pi^+\pi^-\pi^+\pi^0$}{D02Kpipipipi0} decays}
  \begin{itemize}[nosep]
    \setlength{\itemindent}{2.5em}
    \item[\textasteriskcentered]{Explanation of combined mis-ID and missing particle background}
    \item[\textasteriskcentered]{RapidSim studies of mass shape and PID requirements in data}
  \end{itemize}
\subsubsection{Background from \texorpdfstring{$D^0\to K^-\pi^+\pi^-\pi^+$}{D02Kpipipi} decays}
  \begin{itemize}[nosep]
    \setlength{\itemindent}{2.5em}
    \item[\textasteriskcentered]{Present full study of single and triple mis-ID}
  \end{itemize}
\subsection{Invariant mass fits}
  \begin{itemize}[nosep]
    \setlength{\itemindent}{2em}
    \item[\textendash]{Outline the fit strategy}
  \end{itemize}
\subsubsection{Global invariant mass fit}
  \begin{itemize}[nosep]
    \setlength{\itemindent}{2.5em}
    \item[\textasteriskcentered]{Present and describe global mass fit}
    \item[\textasteriskcentered]{Show the same fit, split by charge}
  \end{itemize}
\subsubsection{Binned invariant mass fit}
  \begin{itemize}[nosep]
    \setlength{\itemindent}{2.5em}
    \item[\textasteriskcentered]{Describe binned mass fit}
    \item[\textasteriskcentered]{Show the fitted $C\!P$-violating parameters}
  \end{itemize}
\subsection{Systematic uncertainties}
  \begin{itemize}[nosep]
    \setlength{\itemindent}{2em}
    \item[\textendash]{Very brief description of each systematic uncertainty considered}
  \end{itemize}
\subsubsection{Strong-phase uncertainties}
\subsubsection{Bin-dependent mass shapes}
\subsubsection{Fixed signal shape}
\subsubsection{PID efficiencies}
\subsubsection{Fixed yield fractions}
\subsubsection{Low mass physics effects}
\subsubsection{Small backgrounds not included in the fit}
\subsubsection{Fit bias}
\subsubsection{\texorpdfstring{$D\to K\pi\pi\pi\pi^0$}{D2Kpipipipi0} phase space distribution and mass shape}
\subsubsection{Charmless backgrounds}
\subsubsection{Checks of negligible systematic effects}
\subsubsection{Summary of systematic uncertainties}
\subsection{Interpretation of in terms of \texorpdfstring{$\gamma$}{gamma}}
  \begin{itemize}[nosep]
    \setlength{\itemindent}{2em}
    \item[\textendash]{Combination and interpretation of $C\!P$-violating observables in terms of $\gamma$}
    \item[\textendash]{Highlight the difference between model-dependent and independent results}
    \item[\textendash]{Compare with latest LHCb combination}
  \end{itemize}

\section{The BESIII experiment}
  \begin{itemize}[nosep]
    \setlength{\itemindent}{0em}
    \item{Description of the BESIII experiment and the available data sets}
    \item{Focus on MDC, TOF system and EMC}
  \end{itemize}

\section{Analysis of \texorpdfstring{$D^0\to K^+K^-\pi^+\pi^-$}{D02KKpipi} strong phases}
  \begin{itemize}[nosep]
    \setlength{\itemindent}{0em}
    \item{This section contains the whole BESIII analysis}
    \item{Status: Paper on inclusive phase space measurement submitted to PRD}
    \item{Plan: Binned analysis is ongoing and is expected to finish at the end of TT23}
  \end{itemize}
\subsection{Event selection}
  \begin{itemize}[nosep]
    \setlength{\itemindent}{2em}
    \item[\textendash]{Overview of all the selection requirements}
  \end{itemize}
\subsubsection{Particle selection}
  \begin{itemize}[nosep]
    \setlength{\itemindent}{2.5em}
    \item[\textasteriskcentered]{Charged pion and kaon track requirements}
    \item[\textasteriskcentered]{Photon shower requirements}
    \item[\textasteriskcentered]{Neutral composite particles}
  \end{itemize}
\subsubsection{Tag selection}
  \begin{itemize}[nosep]
    \setlength{\itemindent}{2.5em}
    \item[\textasteriskcentered]{Reconstruction of fully and partially reconstructed $D$-mesons}
    \item[\textasteriskcentered]{Final selection of single- and double-tag events}
    \item[\textasteriskcentered]{Quantum correlated phase space reweighting}
  \end{itemize}
\subsection{Yield determinations using mass fits}
  \begin{itemize}[nosep]
    \setlength{\itemindent}{2em}
    \item[\textendash]{Overview of strategy for determining event yields}
  \end{itemize}
\subsubsection{Single tag yields}
  \begin{itemize}[nosep]
    \setlength{\itemindent}{2.5em}
    \item[\textasteriskcentered]{Description of mass shapes}
    \item[\textasteriskcentered]{Extraction of single tag yields}
  \end{itemize}
\subsubsection{Double tag yields}
  \begin{itemize}[nosep]
    \setlength{\itemindent}{2.5em}
    \item[\textasteriskcentered]{Description of mass shapes and fit strategy}
    \item[\textasteriskcentered]{Simultaneous fit of double tag yields}
  \end{itemize}
\subsubsection{Peaking backgrounds}
  \begin{itemize}[nosep]
    \setlength{\itemindent}{2.5em}
    \item[\textasteriskcentered]{Determination of background yield and mass shapes from simulation}
    \item[\textasteriskcentered]{Quantum correlation corrections in double tag yields}
  \end{itemize}
\subsection{Systematic uncertainties}
  \begin{itemize}[nosep]
    \setlength{\itemindent}{2em}
    \item[\textendash]{Very brief description of each systematic uncertainty considered}
  \end{itemize}
\subsubsection{External parameters}
\subsubsection{Model dependence}
\subsubsection{Peaking backgrounds}
\subsubsection{Single tag yields of \texorpdfstring{$K^0_LX$}{KLX} tags}
\subsubsection{Efficiency factorisation}
\subsubsection{\texorpdfstring{$K^0_S$}{KS0} veto}
\subsubsection{Summary of systematic uncertainties}
\subsection{Maximum likelihood fit of strong phases}
  \begin{itemize}[nosep]
    \setlength{\itemindent}{2em}
    \item[\textendash]{Description of likelihood function and fitter}
  \end{itemize}
\subsubsection{Inclusive phase space measurement}
  \begin{itemize}[nosep]
    \setlength{\itemindent}{2.5em}
    \item[\textasteriskcentered]{Present measurement of $F_+$}
  \end{itemize}
\subsubsection{Binned phase space measurement}
  \begin{itemize}[nosep]
    \setlength{\itemindent}{2.5em}
    \item[\textasteriskcentered]{Present fit results for $c_i$ and $s_i$}
    \item[\textasteriskcentered]{Comparison with model prediction}
    \item[\textasteriskcentered]{Evaluation of binning scheme performance and impact on $\gamma$ measurement}
  \end{itemize}

\section{Analysis of TORCH testbeam PID performance}
  \begin{itemize}[nosep]
    \setlength{\itemindent}{0em}
    \item{This section contains the whole TORCH analysis}
    \item{Status: PID likelihood calculation successfully tested on 2018 testbeam data}
    \item{Plan: Analysis of PID performance using 2022 testbeam data is ongoing and expected to finish at the end of 2023}
  \end{itemize}
\subsection{The TORCH detector}
  \begin{itemize}[nosep]
    \setlength{\itemindent}{2em}
    \item[\textendash]{Describe the design, physics and strategy of TORCH}
  \end{itemize}
\subsubsection{Physics behind TORCH}
  \begin{itemize}[nosep]
    \setlength{\itemindent}{2.5em}
    \item[\textasteriskcentered]{Motivate the need for low momentum PID at LHCb}
    \item[\textasteriskcentered]{Explain how TORCH combines timing and Cherenkov information}
    \item[\textasteriskcentered]{Show the design of TORCH}
    \item[\textasteriskcentered]{Briefly introduce previous studies of timing performance}
  \end{itemize}
\subsection{PID likelihood calculation}
  \begin{itemize}[nosep]
    \setlength{\itemindent}{2em}
    \item[\textendash]{Describe analytical likelihood calculation}
  \end{itemize}
\subsection{November 2022 testbeam}
  \begin{itemize}[nosep]
    \setlength{\itemindent}{2em}
    \item[\textendash]{Briefly mention the motivation and goals of this testbeam campaign}
  \end{itemize}
\subsubsection{Testbeam setup}
  \begin{itemize}[nosep]
    \setlength{\itemindent}{2.5em}
    \item[\textasteriskcentered]{Proto-TORCH}
    \item[\textasteriskcentered]{The Proton Synchrotron T9 beam line}
    \item[\textasteriskcentered]{Cherenkov counters}
    \item[\textasteriskcentered]{Large scintillators}
    \item[\textasteriskcentered]{Timing stations}
    \item[\textasteriskcentered]{Beam telescope}
    \item[\textasteriskcentered]{Trigger Logic Unit}
  \end{itemize}
\subsubsection{PID study of testbeam}
  \begin{itemize}[nosep]
    \setlength{\itemindent}{2.5em}
    \item[\textasteriskcentered]{Calibration of timing information}
    \item[\textasteriskcentered]{Cable length measurement}
    \item[\textasteriskcentered]{PID separation power at different momenta}
    \item[\textasteriskcentered]{PID separation power in different beam positions}
    \item[\textasteriskcentered]{Comparison of PID seperation power with and witout knowledge of $t_0$}
    \item[\textasteriskcentered]{Overall improvement in PID performance at LHCb}
  \end{itemize}

\section{Summary and conclusion}
  \begin{itemize}[nosep]
    \setlength{\itemindent}{0em}
    \item{Summarise LHCb, BESIII and TORCH analyses}
    \item{Outlook on future improvements and measurements}
  \end{itemize}
%%%%%%%%%%%%%%%%%%%%%%%%%%%%%%%%%%%%%%%%%%%%%%%%%%%%%%%%%%%%%%%
\end{document}